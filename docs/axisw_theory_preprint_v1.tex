\documentclass[12pt]{article}
\usepackage{geometry}
\geometry{a4paper, margin=1in}
\usepackage{graphicx}
\usepackage{amsmath}
\usepackage{amsfonts}
\usepackage{amssymb}
\usepackage{hyperref}

\title{Axis{W}: A Hypothetical Framework for Temporal Perception as Displacement in a Fourth Spatial Dimension}

\author{
  Gwen Mesmacre\thanks{ORCID: \href{https://orcid.org/0009-0003-1306-7036}{0009-0003-1306-7036}}\\
  Independent Researcher\\
  \texttt{gwen.mesmacre@proton.me}
}

\date{\today}

\begin{document}

\maketitle

\begin{abstract}
This paper proposes Axis\_W, a theoretical model in which time perception arises from displacement along a fourth spatial axis (W). We present simulations that demonstrate temporal inversion near supermassive black holes. These results offer a novel explanation for observed astrophysical anomalies such as variable redshifts and gamma-ray fluctuations.
\end{abstract}

\section{Introduction}

We perceive time as a linear and universal flow, but this perception may be an emergent consequence of motion through an additional spatial dimension, denoted \( W \). In the proposed Axis\_W model, time is not a fundamental dimension but rather a projection of displacement along this orthogonal axis, relative to the three known spatial coordinates \( (X, Y, Z) \). This framework challenges the standard view of time as a background parameter and instead treats it as a derived quantity shaped by intrinsic motion, local gravitational conditions, and cosmic torsion.

By simulating entities with motion in \( W \), we investigate how perceived time may dilate, invert, or oscillate under different conditions. This approach offers a new interpretation of phenomena typically attributed to relativistic time dilation, potentially explaining astrophysical anomalies that remain elusive under general relativity alone.


\section{Methods}

\subsection{Axis\_W Temporal Framework}
We propose that perceived time, $T_p$, is a function of displacement in a fourth spatial axis $W$. The foundational equation of Axis\_W is:

\[
T_p = \int \left( v_W \cdot \frac{1}{g} + \omega_W \right) \, dt
\]

Where:

\begin{itemize}
\item $T_p$ is the perceived time (accumulated through displacement in $W$),
\item $v_W$ is the intrinsic velocity of the entity in $W$,
\item $g$ is the local gravitational factor (higher $g$ slows time),
\item $\omega_W$ is the global torsion of the universe in $W$,
\item $t$ is the absolute simulation time.
\end{itemize}

In the simulation engine, this integral is approximated stepwise via:

\[
\Delta W(t) = \left( v_W(t) \cdot \frac{1}{g} \right) + \omega_W
\quad \text{and} \quad
T_p(t + \delta t) = T_p(t) + \Delta W(t) \cdot \delta t
\]

This numerical implementation allows for the modeling of time dilation and inversion effects across complex gravitational and torsional fields.

\subsection{Simulation Design}
A Python engine simulates entities evolving in W, using modules for universe dynamics and entity-specific parameters. Data is logged for analysis.

\subsection{Parameters}
\begin{itemize}
\item $\omega_W = \pm 0.05$ rad/s
\item $g=500$ for S-star\_W
\item $\Delta t=1$ s, 20 steps
\end{itemize}

\section{Results and Discussion}

The simulation of S-star\_W near Sagittarius A* revealed:

\begin{center}
\begin{tabular}{|c|c|c|}
\hline
Step & Universe\_W & S-star\_W Time Perceived \\
\hline
1 & -0.05 & -0.0484 \\
2 & -0.1  & -0.0968 \\
... & ... & ... \\
20 & -1.0 & -0.968 \\
\hline
\end{tabular}
\end{center}

All $\Delta W < 0$, implying a retrograde temporal flow. Observational correlations include:
\begin{itemize}
\item M87* brightness variability (EHT, 2019)
\item Absence of pulsars near Sgr A* (Chandra)
\item Gamma-ray flares (Fermi Telescope)
\end{itemize}

\section{Conclusion}

Axis\_W predicts phenomena beyond general relativity: temporal inversion and oscillations in W. These offer testable hypotheses with upcoming observatories.

\subsection*{Future Work}
\begin{itemize}
\item Correlate W dynamics with high-resolution light curves
\item Expand model to galactic and cosmological scales
\end{itemize}

\section*{References}
\begin{thebibliography}{9}
\bibitem{eht2019}
Event Horizon Telescope Collaboration et al., \emph{First M87 Event Horizon Telescope Results}, Astrophysical Journal Letters (2019).

\bibitem{chandra2020}
Chandra X-Ray Observatory, \emph{Sagittarius A* Pulsar Search Results}, NASA (2020).

\bibitem{generalrelativity}
Einstein, A., \emph{The Foundation of the General Theory of Relativity}, Annalen der Physik (1916).

\bibitem{axisw}
Gwen Mesmacre, GPT-4o, \emph{Axis\_W: A Hypothetical Torsional Model of Perceived Time}, Preprint (2025).
\end{thebibliography}

\section*{License}
This project is licensed under the CC BY--NC 4.0 License:  
\url{https://creativecommons.org/licenses/by-nc/4.0/}

You are free to copy, share, and adapt the materials, provided that:
\begin{itemize}
\item Appropriate credit is given to the author.
\item The materials are not used for commercial purposes without permission.
\end{itemize}

For full license details, see: \url{https://creativecommons.org/licenses/by-nc/4.0/}

\copyright{} 2025 Gwen Mesmacre

\end{document}

